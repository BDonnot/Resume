\documentclass[10pt,a4paper]{moderncv}

\moderncvtheme[blue]{classic} 
\usepackage[utf8]{inputenc}
\usepackage[english]{babel}
\usepackage[top=1.1cm, bottom=1.1cm, left=2cm, right=2cm]{geometry}
\usepackage{fontawesome}
%\moderncvstyle{casual}
%\usepackage[scale=0.75]{geometry}

%\includegraphics[height=9pt]{github}}\includegraphics[height=9pt]{linkedin}}\ }

 
%\makeatletter
%
%\renewcommand*{\makecvtitle}{%
%  % recompute lengths (in case we are switching from letter to resume, or vice versa)
%  \recomputecvlengths%
%  % optional detailed information (pre-rendering)
%  \def\phonesdetails{}%
%  \collectionloop{phones}{% the key holds the phone type (=symbol command prefix), the item holds the number
%    \protected@edef\phonesdetails{\phonesdetails\protect\makenewline\csname\collectionloopkey phonesymbol\endcsname\collectionloopitem}}%
%  \def\socialsdetails{}%
%  \collectionloop{socials}{% the key holds the social type (=symbol command prefix), the item holds the link
%    \protected@edef\socialsdetails{\socialsdetails\protect\makenewline\csname\collectionloopkey socialsymbol\endcsname\collectionloopitem}}%
%  \newbox{\makecvtitledetailsbox}%
%  \savebox{\makecvtitledetailsbox}{%
%    \addressfont\color{color2}%
%    \begin{tabular}[b]{@{}r@{}}%
%      \ifthenelse{\isundefined{\@addressstreet}}{}{\makenewline\addresssymbol\@addressstreet%
%        \ifthenelse{\equal{\@addresscity}{}}{}{\makenewline\@addresscity}% if \addresstreet is defined, \addresscity and addresscountry will always be defined but could be empty
%        \ifthenelse{\equal{\@addresscountry}{}}{}{\makenewline\@addresscountry}}%
%      \phonesdetails% needs to be pre-rendered as loops and tabulars seem to conflict
%      \ifthenelse{\isundefined{\@email}}{}{\makenewline\emailsymbol\emaillink{\@email}}%
%      \ifthenelse{\isundefined{\@homepage}}{}{\makenewline\homepagesymbol\httplink{\@homepage}}%
%      \socialsdetails% needs to be pre-rendered as loops and tabulars seem to conflict
%      \ifthenelse{\isundefined{\@extrainfo}}{}{\makenewline\@extrainfo}%
%    \end{tabular}
%  }%
%  % optional photo (pre-rendering)
%  \newbox{\makecvtitlepicturebox}%
%  \savebox{\makecvtitlepicturebox}{%
%    \ifthenelse{\isundefined{\@photo}}%
%    {}%
%    {%
%      \hspace*{\separatorcolumnwidth}%
%      \color{color1}%
%      \setlength{\fboxrule}{\@photoframewidth}%
%      \ifdim\@photoframewidth=0pt%
%        \setlength{\fboxsep}{0pt}\fi%
%      \framebox{\includegraphics[width=\@photowidth]{\@photo}}}}%
%  % name and title
%  \newlength{\makecvtitledetailswidth}\settowidth{\makecvtitledetailswidth}{\usebox{\makecvtitledetailsbox}}%
%  \newlength{\makecvtitlepicturewidth}\settowidth{\makecvtitlepicturewidth}{\usebox{\makecvtitlepicturebox}}%
%  \ifthenelse{\lengthtest{\makecvtitlenamewidth=0pt}}% check for dummy value (equivalent to \ifdim\makecvtitlenamewidth=0pt)
%    {\setlength{\makecvtitlenamewidth}{\textwidth-\makecvtitledetailswidth-\makecvtitlepicturewidth}}%
%    {}%
%  \begin{minipage}[b]{\makecvtitlenamewidth}%
%    \namestyle{\@firstname\ \@lastname}%
%    \ifthenelse{\equal{\@title}{}}{}{\\[1.25em]\titlestyle{\@title}}%
%  \end{minipage}%
%  \hfill%
%  % optional detailed information (rendering)
%  \llap{\usebox{\makecvtitledetailsbox}}% \llap is used to suppress the width of the box, allowing overlap if the value of makecvtitlenamewidth is forced
%  % optional photo (rendering)
%  \usebox{\makecvtitlepicturebox}\\[2.5em]%
%  % optional quote
%  \ifthenelse{\isundefined{\@quote}}%
%    {}%
%    {{\centering\begin{minipage}{\quotewidth}\centering\quotestyle{\@quote}\end{minipage}\\[2.5em]}}%
%  \par
%}% to avoid weird spacing bug at the first section if no blank line is left after \makecvtitle
%\makeatother

\setlength{\hintscolumnwidth}{3cm}

%\photo[3cm]{BDonnot.jpg}
\firstname{Benjamin \\}
\familyname{DONNOT}           
\address{65 avenue de Paris}{92~320 Ch\^atillon FRANCE}   
\mobile{+33~6~12~08~13~43}                          
\email{benjamin.donnot @ gmail.com}
%\defincolor{ForestGreen}
\moderncvcolor{green}  %couleur du thème au choix : blue, orange, green, red, purple, grey, black         
%\linkedin{http://co.linkedin.com/in/johndoe}
%\github{http://github.com/jdoe}
\extrainfo{%
   \faLinkedin ~ \httplink{www.linkedin.com/in/benjamindonnot/} \\
  \faGithub ~ \httplink{github.com/BDonnot}}
  
%\cventry{année--année}{Diplôme}{École}{Ville}{\textit{Mention}}{Description}  % les arguments 3 à 6 peuvent rester vides
%\cventry{année--année}{Poste occupé}{Employeur}{Ville}{}{Courte description des missions, 1 à 2 lignes.}

\nopagenumbers{}                         
\begin{document}
\maketitle

\vspace*{-1cm} 
%{\large \textcolor{color2}{Objectif : recherche d'un CDI/stage de fin d'études en tant que \textit{Datascientist} à partir de \textit{mai 2015} (possibilité éventuelle de commencer à temps partiel à partir de février 2015).}}
%\vspace*{-1cm} 
\section{Work Experience}
\cventry{2019-Now}{Data Scientist}{Paris}{RTE France}{}{My missions include trying to apply state of the art "machine learning" algorithm to real time grid operations such as optimization of topologies or dynamic line rating.}

\cventry{2016-2017}{Assistant professor}{Paris}{ENSAE (\'Ecole Nationale de la Statistique et de l'Administration \'Economique) \& ENPC (\'Ecole Nationale des Ponts et Chaussées)}{}{Assistant professor for the courses of "Introduction to R software", "Python for a Data Scientist" and following student in the case of a last year informatic project (ENSAE), and  "Machine learning and application" (ENPC) }

\cventry{Janvier 2015 \\ \textit{(6 months)}}{\textit{Data Scientist} Intern}{\href{http://www.rte-france.com/}{RTE France}}{Versailles, France}{}{Development and study (in c++ and R) of machine learning algorithm (mainly deep learning and reinforcement learning) in order to better operate the french power grid.}
%Développement en {\textit{R} \& C++} de méthodes d'apprentissage automatique pour la conduite du réseau électrique français.}%Antoine MAROT

\cventry{Février 2015 \\ \textit{(7 months)}}{Consultant for l'ODJ}{\href{http://www.economie.gouv.fr/observatoire-des-jeux}{Observatoire Des Jeux (ODJ)}}{Paris, France}{}{Mission obtained through ENSAE J\textit{unior} \'Etudes. The objective was to build and study \textit{a priori}, different profil of players in oder to predict which one are possibly addicted.}
%Dans le cadre de la Junior Entreprise de l'ENSAE : application de méthodes de "machines learning" (classification) afin de prédire \textit{a priori} quels étaient les profils de joueurs "à risques" à partir d'une base de données contenant des activités de jeu (langage utilisé : \textit{R}).}%Jean-Michel COSTES

\cventry{Février 2014 \\ \textit{(6 months)}}{Stagiaire \textit{Data Scientist}}{\href{http://www.amadeus.com/web/amadeus/fr_FR-FR/Page-daccueil-Amadeus-Home/A-propos-dAmadeus/Notre-entreprise/1259071476540-Page-AMAD_DetailPpal}{Amadeus IT group}}{Sophia-Antipolis, France}{}{Application of machine learning algorithm (classification, clustering, prediction) for predicting the flow of passengers in any airport. Usage of \textit{R} and \textit{Python}}
%Développement en {\textit{R} et \textit{Python}} d'une méthode générique de prévision des flux de passagers dans les aéroports.}%Xavier CALLENS


%\cventry{Août 2013 \\\textit{(6 months)}}{Quantitative Analyst Intern}{\href{http://www.hsbc.fr/1/2/hsbc-france/particuliers}{HSBC}}{Paris, France}{}{Work on local volatility surface and then on jumps (both on spot and volatility).}
%Travaux sur des modèles à volatilité locale puis des modèles à sauts (volatilité et spot) sur les dérivés actions (\textit{R} pour les test, \textit{\textit{C++}} pour l'intégration en production).}%Olivier DUPONT, Nicolas GRANCHAMPS

\cventry{Mars 2013 \\ \textit{(3 months)}}{Consultant}{\href{http://www.jouve.com/fr}{Jouve}}{Paris}{}{Mission obtained through ENSAE J\textit{unior} \'Etudes : detection et rationalization of bids searching, \textit{web-scraping}, supervised clustering of more than $2,5$ millions of bids. This study was made using \textit{Python}.}%Guillaume Arnal

%\cventry{Juillet 2012 \\ \textit{(2 months)}}{"Data base" intern}{\href{https://www.creditmutuel.fr/groupe/fr/index.html}{Confédération Nationale du Crédit Mutuel}}{Paris, France}{}{Writing of synthesis paper, study the customer/bank relationship. Usage of SQL and Excel.}
%Stage d'ouverture au monde professionnel : rédaction de notes synthétiques, étude sur la relation banque/client. Utilisation de SQL puis de Excel.} %Yves CAUHAPE

\section{Education}
\cventry{2015-2019}{PhD in Machine Learning}{CIFRE agreement (Industrial agreement of learning/training by research) between \href{https://www.lri.fr/presentation_en.php}{LRI} lab (\href{https://www.inria.fr/en/}{INRIA}) and \href{http://www.rte-france.com/en}{RTE France} (French Transmission System Operator)}{Paris Area, France}{}{PhD entitled "Deep learning methods for predicting flows in power grids : novel architectures and algorithms" available at \href{https://theses.hal.science/tel-02045873}{https://theses.hal.science/tel-02045873}.}

\cventry{2011-2015}{Student in Data Science, speciality \textit{Machine Learning}}{\href{http://www.ensae.fr/}{ENSAE (Paris Graduate School of Economics, Statistics and Finance), Top French engineer school in statistics}}{Malakoff, Paris Area, France}{}{Main classes : data mining, machine learning, bayesian statistic, computer (C++, Python, R). Full-year internship in 2013-2014}

%\cventry{2012-2013}{1\textsuperscript{st} \& 2\textsuperscript{nd} year applied mathematics}{\href{http://www.ensae.fr/}{ENSAE}}{}{}{Main classes : statistics, econometry, probability, finance, computer (C++, Python, R \& SAS).}

%\cventry{2011-2012}{1\iere{} année mathématiques appliquées}{\href{http://www.ensae.fr/}{ENSAE}}{}{}{Principaux cours : statistiques, économie, informatique (Python, R, SAS).}

\cventry{2008-2011}{CPGE}{\href{http://www4.ac-nancy-metz.fr/lyc-henri-poincare/}{Henri Poincaré Highschool}}{Nancy,France}{Two or Three-years intensive program in mathematics and physics preparing for the national competitive exams for entry to engineering schools.}{}

%

\section{Computer skills}
\cventry{Lead developer}{\href{https://grid2op.readthedocs.io/}{grid2op}, \href{https://lightsim2grid.readthedocs.io/en/latest/}{lightsim2grid}, \href{https://l2rpn-baselines.readthedocs.io/en/latest/}{l2rpn-baselines}}{}{}{}{And such packages all towards}

\cventry{Languages}{Objects Oriented (\textit{C++} -advanced knowledge) \& Functionnal (\textit{OCaml} -knowledge)}{}{}{}{}

\cventry{Statistical}{\textit{Python} (Tensorflow, Scikit-Learn, Pandas, Boost-Python) -advanced knowledge \& \textit{R} (Rcpp, data.table)-advanced knowledge}{}{}{}{}

\cventry{Others}{\LaTeX ; \textit{Pack Office}(VBA, Excel) \& SQL -knowledge}{}{}{}{}

\section{Extra curricular activities}
\cventry{2018-now}{Co organizer of \href{https://l2rpn.chalearn.org/}{l2rpn} competitions}{}{}{}{This serie of competitions aims at democratizing the topic of real time grid operations. By proposing a problem related sequential decision making in power system we hope to help the decarbonization of the powergrid.}
% \cventry{2015}{Co-organizer of \href{http://www.datasciencegame.com/}{"Data Science Game 2015"}}{}{}{}{This is an international students challenge. The students competitors were asked to solve a real data-scientist problem, given by Google. The main sponsors were \textit{Google Inc.} and \textit{Capgemini Consulting}. }
%Il s'agit d'un challenge international étudiant sur le thème de la modélisation dans le cadre de données massives (\textit{"Big Data"}), parrainé par \textit{Google Inc.} et \href{https://www.capgemini.com/}{\textit{Capgemini}}.}

%\cventry{2012-2013 \\ \textit{(14 months)}}{vice treasurer of ENSAE J\textit{unior} \'Etudes}{}{}{}{This is the "junior enterprise" of ENSAE. It is a local non-profit organization entirely managed by students offering consulting services to the corporate. It allows students to work on real professional projects as well as managing an entreprise.}%experiencing unique learning opportunities by doing professional project work on the one side and managing small- to medium-sized enterprises on the other}
%Il s'agit de la junior entreprise de l'ENSAE dont le but est de permettre aux étudiants de mettre en pratique les enseignements reçus à l'ENSAE en réalisant des études confiés par des clients.}
%\cventry{2012-2013 \\ \textit{(10 mois)}}{Co-fondateur d'\textit{ENSAE Finance et Investissements} }{}{}{}{L'association a entre autre pour but de rapprocher le monde académique de celui de la finance par l'organisation de conférences ou "trading games".}

\cventry{}{Sports}{ swimming $\approx$, biking, running (marathon) and occasional hiking and skiing (during vacation)}{}{}{}

%\cventry{}{Randonnée et vélo en montagne}{}{}{}{}

\end{document}
