\documentclass[10pt,a4paper]{moderncv}

\moderncvtheme[blue]{classic} 
\usepackage[utf8]{inputenc}
\usepackage[french]{babel}
\usepackage[top=1.1cm, bottom=1.1cm, left=2cm, right=2cm]{geometry}
\usepackage{fontawesome}
%\makeatletter
%
%\renewcommand*{\makecvtitle}{%
%  % recompute lengths (in case we are switching from letter to resume, or vice versa)
%  \recomputecvlengths%
%  % optional detailed information (pre-rendering)
%  \def\phonesdetails{}%
%  \collectionloop{phones}{% the key holds the phone type (=symbol command prefix), the item holds the number
%    \protected@edef\phonesdetails{\phonesdetails\protect\makenewline\csname\collectionloopkey phonesymbol\endcsname\collectionloopitem}}%
%  \def\socialsdetails{}%
%  \collectionloop{socials}{% the key holds the social type (=symbol command prefix), the item holds the link
%    \protected@edef\socialsdetails{\socialsdetails\protect\makenewline\csname\collectionloopkey socialsymbol\endcsname\collectionloopitem}}%
%  \newbox{\makecvtitledetailsbox}%
%  \savebox{\makecvtitledetailsbox}{%
%    \addressfont\color{color2}%
%    \begin{tabular}[b]{@{}r@{}}%
%      \ifthenelse{\isundefined{\@addressstreet}}{}{\makenewline\addresssymbol\@addressstreet%
%        \ifthenelse{\equal{\@addresscity}{}}{}{\makenewline\@addresscity}% if \addresstreet is defined, \addresscity and addresscountry will always be defined but could be empty
%        \ifthenelse{\equal{\@addresscountry}{}}{}{\makenewline\@addresscountry}}%
%      \phonesdetails% needs to be pre-rendered as loops and tabulars seem to conflict
%      \ifthenelse{\isundefined{\@email}}{}{\makenewline\emailsymbol\emaillink{\@email}}%
%      \ifthenelse{\isundefined{\@homepage}}{}{\makenewline\homepagesymbol\httplink{\@homepage}}%
%      \socialsdetails% needs to be pre-rendered as loops and tabulars seem to conflict
%      \ifthenelse{\isundefined{\@extrainfo}}{}{\makenewline\@extrainfo}%
%    \end{tabular}
%  }%
%  % optional photo (pre-rendering)
%  \newbox{\makecvtitlepicturebox}%
%  \savebox{\makecvtitlepicturebox}{%
%    \ifthenelse{\isundefined{\@photo}}%
%    {}%
%    {%
%      \hspace*{\separatorcolumnwidth}%
%      \color{color1}%
%      \setlength{\fboxrule}{\@photoframewidth}%
%      \ifdim\@photoframewidth=0pt%
%        \setlength{\fboxsep}{0pt}\fi%
%      \framebox{\includegraphics[width=\@photowidth]{\@photo}}}}%
%  % name and title
%  \newlength{\makecvtitledetailswidth}\settowidth{\makecvtitledetailswidth}{\usebox{\makecvtitledetailsbox}}%
%  \newlength{\makecvtitlepicturewidth}\settowidth{\makecvtitlepicturewidth}{\usebox{\makecvtitlepicturebox}}%
%  \ifthenelse{\lengthtest{\makecvtitlenamewidth=0pt}}% check for dummy value (equivalent to \ifdim\makecvtitlenamewidth=0pt)
%    {\setlength{\makecvtitlenamewidth}{\textwidth-\makecvtitledetailswidth-\makecvtitlepicturewidth}}%
%    {}%
%  \begin{minipage}[b]{\makecvtitlenamewidth}%
%    \namestyle{\@firstname\ \@lastname}%
%    \ifthenelse{\equal{\@title}{}}{}{\\[1.25em]\titlestyle{\@title}}%
%  \end{minipage}%
%  \hfill%
%  % optional detailed information (rendering)
%  \llap{\usebox{\makecvtitledetailsbox}}% \llap is used to suppress the width of the box, allowing overlap if the value of makecvtitlenamewidth is forced
%  % optional photo (rendering)
%  \usebox{\makecvtitlepicturebox}\\[2.5em]%
%  % optional quote
%  \ifthenelse{\isundefined{\@quote}}%
%    {}%
%    {{\centering\begin{minipage}{\quotewidth}\centering\quotestyle{\@quote}\end{minipage}\\[2.5em]}}%
%  \par
%}% to avoid weird spacing bug at the first section if no blank line is left after \makecvtitle
%\makeatother

\setlength{\hintscolumnwidth}{3cm}

%\photo[3cm]{BDonnot.jpg}
\firstname{Benjamin \\}
\familyname{DONNOT}           
\address{65 avenue de Paris}{92~320 Châtillon FRANCE}   
\mobile{+33~6~12~08~13~43}                          
\email{benjamin.donnot\at gmail.com}
%\defincolor{ForestGreen}
\moderncvcolor{green}  %couleur du thème au choix : blue, orange, green, red, purple, grey, black         
\extrainfo{%
   \faLinkedin ~ \httplink{www.linkedin.com/in/benjamindonnot/} \\
  \faGithub ~ \httplink{github.com/BDonnot}}
%\cventry{année--année}{Diplôme}{École}{Ville}{\textit{Mention}}{Description}  % les arguments 3 à 6 peuvent rester vides
%\cventry{année--année}{Poste occupé}{Employeur}{Ville}{}{Courte description des missions, 1 à 2 lignes.}

\nopagenumbers{}                         
\begin{document}
\maketitle

\vspace*{-1cm} 
%{\large \textcolor{color2}{Objectif : recherche d'un CDI/stage de fin d'études en tant que \textit{Datascientist} à partir de \textit{mai 2015} (possibilité éventuelle de commencer à temps partiel à partir de février 2015).}}
%\vspace*{-1cm} 

\section{Formation}
\cventry{2015-Aujourd'hui\\ \textit{(3 ans)}}{Thèse CIFRE (en entreprise) : "Méthodes d'apprentissage pour une conduite efficace du réseau électrique"}{\href{https://www.lri.fr/}{INRIA - laboratoire LRI} \& \href{http://www.rte-france.com/}{RTE France} }{Palaiseau/Versailles}{}{Cette thèse vise, à partir d'un historique de situations de réseaux, à trouver les meilleures actions à mettre en place pour garantir la sûreté du réseau électrique (utilisation de \textit{C++}, \textit{R} et \textit{Python}, apprentissage par renforcement et apprentissage profond).}

\cventry{2011-2015}{Formation dans la voie "Data Science" module \textit{Statistique et Apprentissage}}{\href{http://www.ensae.fr/}{ENSAE (\'Ecole Nationale de la Statistique et de l'Administration \'Economique)}}{Malakoff}{}{Principaux cours : data mining, apprentissage statistique, statistiques bayésiennes, informatique (C++, Python, R). Césure en 2013-2014.}

%\cventry{2011-2013}{1\iere{} \& 2\ieme{} année mathématiques appliquées}{\href{http://www.ensae.fr/}{ENSAE}}{}{}{Principaux cours : statistiques, économétrie, probabilités, informatique (C++, Python, R).}%,finance

%\cventry{2011-2012}{1\iere{} année mathématiques appliquées}{\href{http://www.ensae.fr/}{ENSAE}}{}{}{Principaux cours : statistiques, économie, informatique (Python, R, SAS).}

\cventry{2008-2011}{Classe Préparatoire aux Grandes \'Ecoles \-- MPSI puis MP$^*$}{\href{http://www4.ac-nancy-metz.fr/lyc-henri-poincare/}{Lycée Henri Poincaré}}{Nancy}{}{}

\section{Expériences Professionnelles}

\cventry{2016-2017}{Chargé de TDs}{Paris}{ENSAE (\'Ecole Nationale de la Statistique et de l'Administration \'Economique) \& ENPC (\'Ecole Nationale des Ponts et Chaussées)}{}{TD assuré dans les cours "d'Introduction au langage R", de "Python pour un \textit{Data Scientist}" et "Suivi de projet informatique" (ENSAE), ainsi que "Apprentissage Statistique et Applications" (ENPC) }

\cventry{Janvier 2015 \\ \textit{(6 mois)}}{Stagiaire \textit{Data Scientist}}{\href{http://www.rte-france.com/}{RTE France}}{Versailles}{}{Développement en {\textit{R} \& C++} de méthodes d'apprentissage automatique pour la conduite du réseau électrique français: apprentissage profond (\textit{deep learning}) et par renforcement (\textit{reinforcement learning})}%Antoine MAROT

\cventry{Février 2015 \\ \textit{(7 mois)}}{Consultant pour l'ODJ}{\href{http://www.economie.gouv.fr/observatoire-des-jeux}{Observatoire Des Jeux (ODJ)}}{Paris}{}{Dans le cadre de la Junior Entreprise de l'ENSAE : application de méthodes de "machine learning" (classification) afin de prédire \textit{a priori} quels étaient les profils de joueurs "à risques" (vis-à-vis de la dépendance) à partir d'une base de données contenant des activités de jeu (langage utilisé : \textit{R}).}%Jean-Michel COSTES

\cventry{Février 2014 \\ \textit{(6 mois)}}{Stagiaire \textit{Data Scientist}}{\href{http://www.amadeus.com/web/amadeus/fr_FR-FR/Page-daccueil-Amadeus-Home/A-propos-dAmadeus/Notre-entreprise/1259071476540-Page-AMAD_DetailPpal}{Amadeus IT group}}{Sophia-Antipolis}{}{Deuxième stage de césure. Développement en {\textit{R} et \textit{Python}} d'une méthode de prévision des flux de passagers dans les aéroports.}%Xavier CALLENS


\cventry{Août 2013 \\\textit{(6 mois)}}{Stagiaire Quant}{\href{http://www.hsbc.fr/1/2/hsbc-france/particuliers}{HSBC}}{Paris}{}{Premier stage de césure. Travaux sur des modèles à volatilité locale} % puis des modèles à sauts (volatilité et spot) sur les dérivés actions (\textit{R} pour les test, \textit{\textit{C++}} pour l'intégration en production).}%Olivier DUPONT, Nicolas GRANCHAMPS

\cventry{Mars 2013 \\ \textit{(3 mois)}}{Prestataire}{\href{http://www.jouve.com/fr}{Jouve}}{Paris}{}{Dans le cadre de la Junior Entreprise : détection et rationalisation de la recherche d'appels d'offres, \textit{web-scraping}, classification supervisée de plus de $2,5$ millions d'appels d'offres : réalisé en \textit{Python} .}%Guillaume Arnal

%\cventry{Juillet 2012 \\ \textit{(2 mois)}}{Stagiaire 'Base de Données'}{\href{https://www.creditmutuel.fr/groupe/fr/index.html}{Confédération Nationale du Crédit Mutuel}}{Paris}{}{Stage d'ouverture au monde professionnel : rédaction de notes synthétiques \& d'études sur la relation banque/client.} %Yves CAUHAPE


\section{Outils Informatiques}
\cventry{Langages}{Orienté Objets (\textit{C++} -niveau avancé) \& Fonctionnel (\textit{OCaml}, \textit{Scala} -connaissances)}{}{}{}{}


\cventry{Statistiques}{\textit{Python} (Tensorflow, Scikit-Learn, Pandas, Boost-Python) -niveau avancé \& \textit{R} (Rcpp, data.table)-niveau avancé}{}{}{}{}

\cventry{Autres}{\LaTeX ; \textit{Pack Office}(VBA, Excel) \& SQL -connaissances}{}{}{}{}

\section{Langues}
\cventry{Anglais}{Niveau avancé}{Score TOEIC 940/990 points (Juin 2013)}{}{}{}

\cventry{Allemand}{Niveau Moyen}{}{}{}{}

\cventry{Italien}{Niveau Débutant}{}{}{}{}


\section{Vie associative / Sports}
\cventry{2015}{Co-organisateur des éditions 2015, 2016 \& 2017 du \href{http://www.datasciencegame.com/}{"Data Science Game"}}{}{}{}{Il s'agit d'un challenge international étudiant sur le thème de la modélisation dans le cadre de données massives (\textit{"Big Data"}).}
%\cventry{2012-2013 \\ \textit{(14 mois)}}{Vice-trésorier d'ENSAE J\textit{unior} \'Etudes}{}{}{}{Il s'agit de la junior entreprise de l'ENSAE dont le but est de permettre aux étudiants de mettre en pratique les enseignements reçus à l'ENSAE en réalisant des études confiés par des clients.}
%\cventry{2012-2013 \\ \textit{(10 mois)}}{Co-fondateur d'\textit{ENSAE Finance et Investissements} }{}{}{}{L'association a entre autre pour but de rapprocher le monde académique de celui de la finance par l'organisation de conférences ou "trading games".}

\cventry{}{Natation, randonnée \& course à pieds}{}{}{}{}

%\cventry{}{Randonnée et vélo en montagne}{}{}{}{}

\end{document}
